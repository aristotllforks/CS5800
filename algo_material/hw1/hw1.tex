\documentclass[11pt]{article}

\newcommand{\yourname}{Zerun Tian}
\newcommand{\yourcollaborators}{}

\def\comments{0}

%format and packages

%\usepackage{algorithm, algorithmic}
\usepackage[noend]{algpseudocode}
\usepackage{algorithm}
\usepackage{amsmath, amssymb, amsthm}
\usepackage{enumerate}
\usepackage{enumitem}
\usepackage{framed}
\usepackage{verbatim}
\usepackage[margin=1.0in]{geometry}
\usepackage{microtype}
\usepackage{kpfonts}
\usepackage{palatino}
	\DeclareMathAlphabet{\mathtt}{OT1}{cmtt}{m}{n}
	\SetMathAlphabet{\mathtt}{bold}{OT1}{cmtt}{bx}{n}
	\DeclareMathAlphabet{\mathsf}{OT1}{cmss}{m}{n}
	\SetMathAlphabet{\mathsf}{bold}{OT1}{cmss}{bx}{n}
	\renewcommand*\ttdefault{cmtt}
	\renewcommand*\sfdefault{cmss}
	\renewcommand{\baselinestretch}{1.06}
\usepackage[usenames,dvipsnames]{xcolor}
\definecolor{DarkGreen}{rgb}{0.15,0.5,0.15}
\definecolor{DarkRed}{rgb}{0.6,0.2,0.2}
\definecolor{DarkBlue}{rgb}{0.2,0.2,0.6}
\definecolor{DarkPurple}{rgb}{0.4,0.2,0.4}
\usepackage[pdftex]{hyperref}
\hypersetup{
	linktocpage=true,
	colorlinks=true,				% false: boxed links; true: colored links
	linkcolor=DarkBlue,		% color of internal links
	citecolor=DarkBlue,	% color of links to bibliography
	urlcolor=DarkBlue,		% color of external links
}


%\usepackage[boxruled,vlined,nofillcomment]{algorithm2e}
%	\SetKwProg{Fn}{Function}{\string:}{}
%	\SetKwFor{While}{While}{}{}
%	\SetKwFor{For}{For}{}{}
%	\SetKwIF{If}{ElseIf}{Else}{If}{:}{ElseIf}{Else}{:}
%	\SetKw{Return}{Return}
	

%enclosure macros
\newcommand{\paren}[1]{\ensuremath{\left( {#1} \right)}}
\newcommand{\bracket}[1]{\ensuremath{\left\{ {#1} \right\}}}
\renewcommand{\sb}[1]{\ensuremath{\left[ {#1} \right\]}}
\newcommand{\ab}[1]{\ensuremath{\left\langle {#1} \right\rangle}}

%probability macros
\newcommand{\ex}[2]{{\ifx&#1& \mathbb{E} \else \underset{#1}{\mathbb{E}} \fi \left[#2\right]}}
\newcommand{\pr}[2]{{\ifx&#1& \mathbb{P} \else \underset{#1}{\mathbb{P}} \fi \left[#2\right]}}
\newcommand{\var}[2]{{\ifx&#1& \mathrm{Var} \else \underset{#1}{\mathrm{Var}} \fi \left[#2\right]}}

%useful CS macros
\newcommand{\poly}{\mathrm{poly}}
\newcommand{\polylog}{\mathrm{polylog}}
\newcommand{\zo}{\{0,1\}}
\newcommand{\pmo}{\{\pm1\}}
\newcommand{\getsr}{\gets_{\mbox{\tiny R}}}
\newcommand{\card}[1]{\left| #1 \right|}
\newcommand{\set}[1]{\left\{#1\right\}}
\newcommand{\negl}{\mathrm{negl}}
\newcommand{\eps}{\varepsilon}
\DeclareMathOperator*{\argmin}{arg\,min}
\DeclareMathOperator*{\argmax}{arg\,max}
\newcommand{\eqand}{\qquad \textrm{and} \qquad}
\newcommand{\ind}[1]{\mathbb{I}\{#1\}}
\newcommand{\sslash}{\ensuremath{\mathbin{/\mkern-3mu/}}}

%mathbb
\newcommand{\N}{\mathbb{N}}
\newcommand{\R}{\mathbb{R}}
\newcommand{\Z}{\mathbb{Z}}
%mathcal
\newcommand{\cA}{\mathcal{A}}
\newcommand{\cB}{\mathcal{B}}
\newcommand{\cC}{\mathcal{C}}
\newcommand{\cD}{\mathcal{D}}
\newcommand{\cE}{\mathcal{E}}
\newcommand{\cF}{\mathcal{F}}
\newcommand{\cL}{\mathcal{L}}
\newcommand{\cM}{\mathcal{M}}
\newcommand{\cO}{\mathcal{O}}
\newcommand{\cP}{\mathcal{P}}
\newcommand{\cQ}{\mathcal{Q}}
\newcommand{\cR}{\mathcal{R}}
\newcommand{\cS}{\mathcal{S}}
\newcommand{\cU}{\mathcal{U}}
\newcommand{\cV}{\mathcal{V}}
\newcommand{\cW}{\mathcal{W}}
\newcommand{\cX}{\mathcal{X}}
\newcommand{\cY}{\mathcal{Y}}
\newcommand{\cZ}{\mathcal{Z}}

%theorem macros
\newtheorem{thm}{Theorem}
\newtheorem{lem}[thm]{Lemma}
\newtheorem{fact}[thm]{Fact}
\newtheorem{clm}[thm]{Claim}
\newtheorem{rem}[thm]{Remark}
\newtheorem{coro}[thm]{Corollary}
\newtheorem{prop}[thm]{Proposition}
\newtheorem{conj}[thm]{Conjecture}

\theoremstyle{definition}
\newtheorem{defn}[thm]{Definition}


\newcommand{\instructor}{Virgil Pavlu}
\newcommand{\hwnum}{1}
\newcommand{\hwdue}{Wednesday, January 27 at 11:59pm via \href{https://gradescope.com/courses/229309}{Gradescope}}

\theoremstyle{theorem}
\newtheorem{prob}{}
\newtheorem{sol}{Solution}

\definecolor{cit}{rgb}{0.05,0.2,0.45} 
\newcommand{\solution}{\medskip\noindent{\color{DarkBlue}\textbf{Solution:}}}

\begin{document}
{\Large 
\begin{center}{CS5800: Algorithms} --- Spring '21 --- \instructor \end{center}}
{\large
\vspace{10pt}
\noindent Homework~\hwnum \vspace{2pt}\\
Due :~\hwdue}

\bigskip
{\large \noindent Name: \yourname }

\vspace{15pt}

{\large \noindent Instructions:}

\begin{itemize}

\item Make sure to put your name on the first page.  If you are using the \LaTeX~template we provided, then you can make sure it appears by filling in the \texttt{yourname} command.

\item Please review the grading policy outlined in the course information page.

\item You must also write down with whom you worked on the assignment.  If this changes from problem to problem, then you should write down this information separately with each problem.

\item Problem numbers (like Exercise 3.1-1) are corresponding to CLRS $3^{rd}$ edition.  While the  $2^{nd}$ edition  has  similar  problems  with  similar  numbers,  the  actual  exercises  and their solutions are different, so make sure you are using the $3^{rd}$ edition.

\end{itemize}

\newpage

\begin{prob} \textbf{(20 points)}
\end{prob}

Two linked lists (simple link, not double link) heads are given:headA, andheadB;it is also given that the two lists intersect,  thus after the intersection they have thesame elements to the end.  Find the first common element, without modifying the listselements or using additional datastructures.

\begin{enumerate}[label=(\alph*)]

\item  A  linear  algorithm  is  discussed  in  the  lecture:  count  the  lists  first,  then  use  the count  difference  as  an  offset  in  the  longer  list,  before  traversing  the  lists  together. Write a formal pseudocode (the pseudocode in the lecture is vague), using “next” as a method/pointer to advance to the next element in a list.

%%% Problem 1.a %%%
\solution

Counting the number of nodes in a linked list (helper function):

\begin{algorithmic}[1]
\Function{CountNodes}{$head$}
	\State $\textit{count} = 0$
	\State $p = head$
	\While{$p$ is not None} 
		\State $\textit{count} = \textit{count} + 1$
		\State $p = p.next$
	\EndWhile
	\State \textbf{return} $\textit{count}$
\EndFunction
\end{algorithmic}


Finding the first intersecting node:

\begin{algorithmic}[1]
\Function{FindIntersection}{$headA$, $headB$}
	\State $lenA = \textproc{\textsc{CountNodes}}(headA)$
	\State $lenB = \textproc{CountNodes}(headB)$
	\State $\textit{offset} = abs(lenA - lenB)$ \Comment{difference in size between the two lists}
	\State $pl = headA$ \Comment{a pointer to the longer list}
	\State $ps = headB$ \Comment{a pointer to the shorter list}
	\If{$lenA < lenB$}
		\State $pl = headB$
		\State $ps = headA$
	\EndIf
	\While{$\textit{offset} > 0$} \Comment{move the pointer of the longer list by offset}
		\State $pl = pl.next$
		\State $\textit{offset} = \textit{offset} - 1$
	\EndWhile
	\While{$pl$ is not None} \Comment{iterate simultaneously to find the intersecting node}
		\If{$pl = ps$}
			\State \textbf{return} $pl$
		\EndIf
		\State $pl = pl.next$
		\State $ps = ps.next$
	\EndWhile
	\State \textbf{return} None \Comment{no intersecting node found}
\EndFunction
\end{algorithmic}



\item Write the actual code in a programming language (C/C++, Java, Python etc) of your  choice  and  run  it  on  a  made-up  test  pair  of  two  lists.   A  good  idea  is  to  use pointers to represent the list linkage.

%%% Problem 1.b %%%
\solution  $\vspace{10 mm}$ Code is listed in the file $\texttt{hw1.py}$

\end{enumerate}

\newpage
\begin{prob} \textbf{(10 points)} Exercise 3.1-1
\end{prob}

%%% Problem 2 %%%
Let $f(n)$ and $g(n)$ be asymptotically nonnegative functions. Using the basic definition of $\Theta$-notation, prove that $\text{max}(f(n), g(n)) = \Theta(f(n) + g(n))$.

\solution

By the definition of big-$\Theta$, we are trying to show that with positive constants $c_1$, $c_2$ and $n_0$,
\[
c_1 \cdot (f(n) + g(n)) \le \text{max}(f(n), g(n)) \le c_2 \cdot (f(n) + g(n))
\]
for $n \ge n_0$.

Since $f(n)$ and $g(n)$ are asymptotically nonnegative, there exists a positive constant $n_f$ s.t. $f(n) \ge 0$ when $n \ge n_f$. Similarly,  $g(n) \ge 0$ when $n \ge n_g$. We know that $f(n) + g(n)$ is nonnegative when $n \ge \text{max}(n_f, n_g)$, which met one of the necessary conditions of big-$\Theta$.

We can derive some other findings for $n \ge \text{max}(n_f, n_g)$,
\[
f(n) \le \text{max}(f(n), g(n))
\]
\[
g(n) \le \text{max}(f(n), g(n))
\]

By combining the two inequalities above, we get $\frac{1}{2}(f(n) + g(n)) \le \text{max}(f(n), g(n))$. 
Moreover, we know $\text{max}(f(n), g(n)) \le f(n) + g(n)$ because they are nonnegative functions.

In conclusion, we find the formula below holds,
\[
\frac{1}{2} \cdot (f(n) + g(n)) \le \text{max}(f(n), g(n)) \le f(n) + g(n)
\]
with $c_1 = \frac{1}{2}$, $c_2 = 1$, and $n_0 = max(n_f, n_g)$, so $\text{max}(f(n), g(n)) = \Theta(f(n) + g(n))$.

\newpage
\begin{prob} \textbf{(5 points)} Exercise 3.1-4
\end{prob}

%%% Problem 3 %%%
Is $2^{n+1} = O(2^n)$? Is $2^{2n} = O(2^n)$?

\solution

The LHS $2^{n+1}$ can be written as $2 \cdot 2^n$. By definition of big-$O$, we want to find a constant $c$ such that  $2 \cdot 2^n \leq c \cdot 2^{n}$. For example, $c = 3$ is a valid choice. We showed $2^{n+1} = O(2^n)$.

The LHS $2^{2n}$ can be written as $(2^2)^n$, which is $4^n$. We attempt to find a constant $c$ such that $4^n \leq c \cdot 2^n$. Let's move the terms to get $\frac{4^n}{2^n} \leq c$. The simplified formula $2^n \leq c$ does not hold as exponential grows much faster than constant. We showed $2^{2n} \neq O(2^n)$.

\newpage
%%% Problem 4 %%%
\begin{prob} \textbf{(15 points)}
\end{prob}

Rank  the  following  functions  in  terms  of  asymptotic  growth.   In  other  words, find an arrangement of the functions $f_1, f_2, . . .$ such that for all i, $f_i = \Omega(f_{i+1})$.

\begin{center}
\begin{tabular}{ccccccc}
$\sqrt{n}\ln n$ & $\ln \ln n^2$ & $2^{\ln^2 n}$ & $n!$ & $n^{0.001}$ & $2^{2\ln n}$  & $(\ln n)!$ \\
\end{tabular}
\end{center}

\solution

In decreasing order of growth rate:

\[
n!   \geq   2^{\ln^2 n}   \geq   (\ln n)!   \geq   2^{2\ln n}   \geq   \sqrt{n}\ln n   \geq n^{0.001}   \geq   \ln \ln n^2
\]

\textbf{Reasoning:} 

\begin{enumerate}
\item Show $n! = \Omega(2^{\ln^2 n})$: \\
Using the Stirling's approximation, $n! \approx \sqrt{2 \pi} n^{n + 1/2} e^{-n}$. Let's simplify the RHS formula $2^{\ln^2 n}$,
\[
2^{\ln^2 n} = (2^{\ln n})^{\ln n} = (n^{\ln 2})^{\ln n}
\]
where $2^{\ln n}$ can be expressed as $n^{\ln 2}$ based on the formula $a^{\log_c^b} = b^{\log_c^a}$ we learned during the first tutoring session. Then, we can show that,
\[
\begin{split}
\lim_{n \to \infty} \frac{n!}{2^{\ln^2 n}} 
\approx \lim_{n \to \infty} \frac{\sqrt{2 \pi} n^{n + 1/2} e^{-n}}{(n^{\ln 2})^{\ln n}} 
&= \lim_{n \to \infty} \frac{\ln n^{n+1/2} - n}{\ln 2 \ln n \ln n} \\
&= \lim_{n \to \infty} \frac{(n+1/2) \ln n - n}{\ln 2 \ln^2 n} \\
&= \lim_{n \to \infty} \frac{n \ln n + (\ln n)/2 - n}{\ln 2 \ln^2 n} \\
&\approx \lim_{n \to \infty} \frac{n}{\ln 2 \ln n} \\
&= \infty
\end{split}
\]

\item Show $2^{\ln^2 n} = \Omega((\ln n)!)$: \\
We have that $(\ln n)! \approx \sqrt{2 \pi} (\ln n)^{\ln n + 1/2} e^{-\ln n}$ using the Stirling's approximation. We take the limit as follows,
\[
\begin{split}
\lim_{n \to \infty} \frac{2^{\ln^2 n}}{(\ln n)!} 
\approx \lim_{n \to \infty} \frac{(n^{\ln 2})^{\ln n}}{\sqrt{2 \pi} (\ln n)^{\ln n + 1/2} e^{-\ln n}} 
& \approx \lim_{n \to \infty} \frac{(n^{\ln 2})^{\ln n}}{(\ln n)^{\ln n + 1/2} e^{-\ln n}} \\
&= \lim_{n \to \infty} \frac{\ln 2 \ln^2 n}{(\ln n + 1/2) \ln \ln n - \ln n} \\
&= \lim_{n \to \infty} \frac{\ln 2 \ln^2 n}{(\ln \ln n) \ln n + (\ln \ln n)/2  - \ln n} \\
& \approx \lim_{n \to \infty} \frac{\ln 2 \ln n}{\ln \ln n  - 1} \\
&= \infty
\end{split}
\]

\item Show $(\ln n)! = \Omega(2^{2\ln n})$: \\
We take the limit as follows,
\[
\begin{split}
\lim_{n \to \infty} \frac{(\ln n)!}{2^{2\ln n}}
\approx \lim_{n \to \infty} \frac{(\ln n)^{\ln n + 1/2} \cdot e^{-\ln n}}{4^{\ln n}}
&= \lim_{n \to \infty} \frac{(\ln n)^{\ln n} \cdot (\ln n)^{1/2} e^{-\ln n}}{4^{\ln n}} \\
&= \lim_{n \to \infty} \frac{(\ln \ln n) \ln n + (\ln \ln n) / 2 -\ln n}{(\ln 4) \ln n} \\
& \approx \lim_{n \to \infty} \frac{(\ln \ln n) - 1}{\ln 4} \\
&= \infty
\end{split}
\]

\item Show $2^{2\ln n} = \Omega(\sqrt{n} \ln n)$: \\
The LHS can be written as $2^{2\ln n} = 4^{\ln n} = n^{\ln 4} \approx n^{1.386}$. We take the limit as follows,
\[
\lim_{n \to \infty} \frac{2^{2\ln n}}{\sqrt{n} \ln n}
\approx \lim_{n \to \infty} \frac{n^{\ln 4}}{n^{0.5} \ln n}
= \lim_{n \to \infty} \frac{n^{((\ln 4) - 0.5)}}{\ln n} = \infty
\]

\item Show $\sqrt{n} \ln n = \Omega(n^{0.001})$: \\
Let's take the limit as follows,
\[
\lim_{n \to \infty} \frac{\sqrt{n} \ln n}{n^{0.001}}
= \lim_{n \to \infty} \frac{n^{0.5} \ln n}{n^{0.001}}
= \lim_{n \to \infty} \frac{n^{0.499} \ln n}{1}
= \infty
\]

\item Show $n^{0.001} = \Omega(\ln \ln n^2)$: \\
We take the limit as follows,
\[
\lim_{n \to \infty} \frac{n^{0.001}}{\ln \ln n^2}
= \lim_{n \to \infty} \frac{n^{0.001}}{\ln (2 \ln n)}
= \lim_{n \to \infty} \frac{n^{0.001}}{\ln 2 + \ln \ln n}
= \infty
\]

\end{enumerate}

\newpage
\begin{prob} \textbf{(40 points)} Problem 4-1 (page 107)
\end{prob}

Give asymptotic upper and lower bounds for T(n) in each of the following recurrences. Assume that T(n) is constant for $n \leq 2$. Make your bounds as tight as possible, and justify your answers.

\begin{enumerate}[label=(\alph*)]

\item $T(n) = 2T(n/2) + n^4$

%%% Problem 5.a %%%
\solution

It has the form $T(n) = aT(n/b) + n^c$, so we use the master method learned in class.

The three cases are: 

- case 1: $\log_b^a > c$, then $T(n) = \Theta(n^{\log_b^a})$

- case 2: $\log_b^a = c$, then $T(n) = \Theta(n^{\log_b^a} \log n)$

- case 3: $\log_b^a < c$, then $T(n) = \Theta(n^c)$

With $a = 2$, $b = 2$, and $c = 4$, we find $\log_b^a = \log_2^2 = 1 < c$.

This matches case 3, so we get that $T(n) = \Theta(n^4)$.

\item $T(n) = T(7n/10) + n$

%%% Problem 5.b %%%
\solution

With $a = 1$, $b = \frac{10}{7}$, and $c = 1$, we find $\log_b^a = \log_{10/7}^{1} = 0 < c$.

This matches case 3, so we get that $T(n) = \Theta(n)$.

\item $T(n) = 16T(n/4) + n^2$

%%% Problem 5.c %%%
\solution

With $a = 16$, $b = 4$, and $c = 2$, we find $\log_b^a = \log_4^{16} = 2 = c$.

This matches case 2, so we get that $T(n) = \Theta(n^2 \log n)$.

\item $T(n) = 7T(n/3) + n^2$

%%% Problem 5.d %%%
\solution

With $a = 7$, $b = 3$, and $c = 2$, we find $\log_b^a = \log_3^7 \approx 1.771 < c$.

This matches case 3, so we get that $T(n) = \Theta(n^2)$.

\item $T(n) = 7T(n/2) + n^2$

%%% Problem 5.e %%%
\solution

With $a = 7$, $b = 2$, and $c = 2$, we find $\log_b^a = \log_2^7 \approx 2.807 > c$.

This matches case 1, so we get that $T(n) = \Theta(n^{\log_{2}^{7}})$.

\item $T(n) = 2T(n/4) + \sqrt{n}$

%%% Problem 5.f %%%
\solution

With $a = 2$, $b = 4$, and $c = 0.5$, we find $\log_b^a = \log_4^2 = 0.5 = c$.

This matches case 2, so we get that $T(n) = \Theta(\sqrt{n} \log n)$.

\item $T(n) = T(n-2) + n^2$

%%% Problem 5.g %%%
\solution

Let's solve this recurrence via iteration.
\[
\begin{split}
T(n) & = T(n-2) + n^2 \\
& = {[T(n-4) + (n-2)^2]} + n^2 = T(n-4) + (n-2)^2 + n^2 \\ 
& = {[T(n-6) + (n-4)^2]} + (n-2)^2 + n^2 = T(n-6) + (n-4)^2 + (n-2)^2 + n^2 \\
&\;\;\vdots \notag \\
& = T(n-2k) + (n - 2(k-1))^2 + \cdots + (n-4)^2 + (n-2)^2 + (n-0)^2 \\
& = T(n-2k) + \sum_{i=0}^{k-1} (n - 2i)^2 
\end{split}
\]
Now, we show this pattern of $k$ is correct, by induction.

\textbf{Claim:} For all $k \geq 1$, $T(n-2k) + \sum_{i=0}^{k-1} (n - 2i)^2$.

\textbf{Proof:} 

- The base case, $k = 1$, is true as the resulting equation $T(n-2) + n^2$ matches the original recurrence.

- Inductive hypothesis: assuming the claim is true for $k = j$. i.e.,
\[
T(n) = T(n-2j) + \sum_{i=0}^{j-1} (n - 2i)^2
\]

- Inductive step: showing the claim holds true for $k = j + 1$. i.e.,
\[
T(n) = T(n-2(j+1)) + \sum_{i=0}^{j} (n - 2i)^2
\]

Let's expand the inductive hypothesis by applying the definition of the recurrence,
\[
\begin{split}
T(n) & = T(n-2j) + \sum_{i=0}^{j-1} (n - 2i)^2 \\
& = [T(n-2j-2) + (n-2j)^2] + \sum_{i=0}^{j-1} (n - 2i)^2 \\
& = T(n-2(j+1)) + (n-2j)^2 + (n-2(j-1))^2 + \cdots + (n-2)^2 + (n-0)^2 \\
& = T(n-2(j+1)) + \sum_{i=0}^{j} (n - 2i)^2
\end{split}
\]

We thus have that $T(n) = T(n-2k) + \sum_{i=0}^{k-1} (n - 2i)^2$ for $k \geq 1$. Now, let's choose a $k$ that would lead to a base case. Given $T(n)$ is constant for $n \leq 2$, we want $n-2k = 2$, then $k = \frac{n}{2} - 1$,

\[
\begin{split}
T(n) & = T(2) + \sum_{i=0}^{n/2-2} (n - 2i)^2 \\
& = T(2) + \sum_{i=0}^{n/2-2} n^2 - 4ni + 4i^2 \\
& = T(2) + n^2 \sum_{i=0}^{n/2-2} 1 - 4n \sum_{i=0}^{n/2-2} i + 4 \sum_{i=0}^{n/2-2} i^2 \\
& = T(2) + n^2 (\frac{n}{2} - 1) - 4n (\frac{1}{2} (1 + \frac{n}{2} - 2) (\frac{n}{2} - 2)) + 4 (\frac{1}{6} (\frac{n}{2} - 2) (\frac{n}{2} - 2 + 1) (n - 4 + 1)) \\
& = T(2) + \frac{n^3}{2} - n^2 - \frac{n^3}{2} + 2n^2 + n^2 - 4n + \frac{n^3}{6} - n^2 + \frac{4n}{3} - \frac{n^2}{2} + 3n - 4 \\
& = T(2) + \frac{n^3}{6} - \frac{n^2}{2} - \frac{n}{3} - 4 \\
& = \Theta(1) + \Theta(n^3) \\
& = \Theta(n^3)
\end{split}
\] 

\end{enumerate}

\newpage
\begin{prob} \textbf{(30 points)} Problem 4-3 from (a) to (f) (page 108)
\end{prob}

Give asymptotic upper and lower bounds for T(n) in each of the following recurrences. Assume that T(n) is constant for sufficiently small n. Make your bounds as tight as possible, and justify your answers.

\begin{enumerate}[label=(\alph*)]

\item $T(n) = 4T(n/3) + n\lg n$

%%% Problem 6.a %%%
\solution

Using the iteration method, we expand the above recurrence as follows,
\[
\begin{split}
T(n) &= 4T(\frac{n}{3}) + n\lg n \\
&= 4[4T(\frac{n}{9}) + \frac{n}{3} \lg \frac{n}{3}] + n \lg n 
= 16 T(\frac{n}{9}) + 4 \frac{n}{3} \lg \frac{n}{3} + n \lg n \\
&= 16[4T(\frac{n}{27}) + \frac{n}{9} \lg \frac{n}{9}] +  4 \frac{n}{3} \lg \frac{n}{3} + n \lg n
= 64 T(\frac{n}{27}) + 4^2 \frac{n}{9} \lg \frac{n}{9} + 4 \frac{n}{3} \lg \frac{n}{3} + n \lg n \\
&= 64[4T(\frac{n}{81}) + \frac{n}{27} \lg \frac{n}{27}] + 4^2 \frac{n}{9} \lg \frac{n}{9} + 4 \frac{n}{3} \lg \frac{n}{3} + n \lg n \\
&= 256T(\frac{n}{81}) + 4^3 \frac{n}{3^3} \lg \frac{n}{3^3} + 4^2 \frac{n}{3^2} \lg \frac{n}{3^2} + 4 \frac{n}{3} \lg \frac{n}{3} + n \lg n 
\end{split}
\]
Now, we can see a clear pattern in terms of k,
\[
T(n) = 4^k T(\frac{n}{3^k}) + \sum_{i=0}^{k-1} 4^i \frac{n}{3^i} \lg \frac{n}{3^i}
\]
We omit proving the pattern's correctness as the form is evident through several iterations. To reach a base case, we want to make $k	\simeq log_3^n$. Plugging $k$ into the recurrence gives us,
\[
\begin{split}
T(n) &= 4^{\log_3^n} T(\frac{n}{3^{\log_3^n}}) + \sum_{i=0}^{\log_3^n-1} 4^i \frac{n}{3^i} \lg \frac{n}{3^i} \\
&= n^{\log_3^4} T(1) + n \sum_{i=0}^{\log_3^n-1} \frac{4^i}{3^i} (\lg n - \lg 3^i) \\
&= n^{\log_3^4} T(1) + n \sum_{i=0}^{\log_3^n-1} \frac{4^i}{3^i} \lg n - n \lg 3 \sum_{i=0}^{\log_3^n-1} i (\frac{4}{3})^i
\end{split}
\]
Let's try to resolve the second term and the third term of the above equation. The second term can be simplified further to,
\[
\begin{split}
n \sum_{i=0}^{\log_3^n-1} \frac{4^i}{3^i} \lg n 
&= n[\lg n + \frac{4}{3} \lg n + \frac{16}{9} \lg n + \cdots + (\frac{4}{3})^{\log_3^n - 1} \lg n] \\
&= n[\lg n \sum_{i=0}^{\log_3^n - 1} (\frac{4}{3})^i] \\
&= n[\lg n (3n^{\log_3^4 - 1} - 3)] \\
&= 3n^{\log_3^4} \lg n - 3n \lg n
\end{split}
\]
Before we proceed on simplifying the third term, we know the closed form of the summation $\sum_{i=0}^n i a^i = \frac{a - a^{n+1}}{(1-a)^2} - \frac{na^{n+1}}{1-a}$. Thus, let's work on the third term,
\[
\begin{split}
n \lg 3 \sum_{i=0}^{\log_3^n-1} i (\frac{4}{3})^i
&= n \lg 3 (\frac{4/3 - (4/3)^{\log_3^n}}{(1-4/3)^2} - \frac{(\log_3^n - 1) (4/3)^{\log_3^n}}{1 - 4/3}) \\
&= n \lg 3 (12 - 9n^{\log_3^4 - 1} + 3n^{\log_3^4 - 1} \log_3^n - 3n^{log_3^4 - 1}) \\
&= (12\lg 3) n - (9\lg 3) n^{\log_3^4} + (3\lg 3) n^{\log_3^4}  \log_3^n - (3\lg 3) n^{\log_3^4}
\end{split}
\]
Let's simplify the term $(3\lg 3) n^{\log_3^4}  \log_3^n$ of the above equation,
\[
\begin{split}
(3\lg 3) n^{\log_3^4}  \log_3^n &= 3 n^{\log_3^4} \lg 3 \log_3^n \\
&= 3 n^{\log_3^4} \log_3^{n^{\lg 3}} \\
&= 3 n^{\log_3^4} \log_3^{3^{\lg n}} \\
&= 3 n^{\log_3^4} \lg n
\end{split}
\]
Combining the closed forms of the two summations, we write the recurrence as,
\[
\begin{split}
T(n) &= n^{\log_3^4} T(1) + n \sum_{i=0}^{\log_3^n-1} \frac{4^i}{3^i} \lg n - n \lg 3 \sum_{i=0}^{\log_3^n-1} i (\frac{4}{3})^i \\
&= n^{\log_3^4} T(1) + [3n^{\log_3^4} \lg n - 3n \lg n] - [(12\lg 3) n - (9\lg 3) n^{\log_3^4} + 3 n^{\log_3^4} \lg n - (3\lg 3) n^{\log_3^4}] \\
&= n^{\log_3^4} T(1) - 3n \lg n - (12\lg 3) n + (9\lg 3) n^{\log_3^4}  + (3\lg 3) n^{\log_3^4} \\
&= [\Theta(1) + 9\lg 3 + 3\lg 3] n^{\log_3^4} - 3n \lg n - (12\lg 3) n \\
&= \Theta(n^{\log_3^4})
\end{split}
\]
Therefore, the asymptotic runtime is $T(n) = \Theta(n^{\log_3^4})$.

\item $T(n) = 3T(n/3) + n / \lg n$

%%% Problem 6.b %%%
\solution

Using the iteration method, we expand the above recurrence as follows,
\[
\begin{split}
T(n) &= 3T(\frac{n}{3}) + n / \lg n \\
&= 3[3T(\frac{n}{9}) + \frac{n}{3}(\lg \frac{n}{3})^{-1}] + n (\lg n)^{-1} = 9T(\frac{n}{9}) + n(\lg \frac{n}{3})^{-1} + n (\lg n)^{-1} \\
& = 9[3T(\frac{n}{27}) + \frac{n}{9}(\lg \frac{n}{9})^{-1}] + n(\lg \frac{n}{3})^{-1} + n (\lg n)^{-1} = 27T(\frac{n}{27}) + n(\lg \frac{n}{9})^{-1} + n(\lg \frac{n}{3})^{-1} + n (\lg n)^{-1}
\end{split}
\]
Now, we can see a clear pattern to represent the recurrence in terms of k,
\[
T(n) = 3^k T(\frac{n}{3^k}) + n \sum_{i=0}^{k-1} (\lg \frac{n}{3^i})^{-1}
\]
We omit proving the correctness of the pattern via induction because its form is evident.
Then, we choose $k = \log_3^n$, which leads to a base case.
\[
\begin{split}
T(n) &= 3^{\log_3^n} T(\frac{n}{3^{\log_3^n}}) + n \sum_{i=0}^{\log_3^n -1} (\lg \frac{n}{3^i})^{-1} \\
&=  n T(1) + n [(\lg n)^{-1} + (\lg \frac{n}{3})^{-1} + (\lg \frac{n}{9})^{-1} +  \cdots + (\lg \frac{n}{3^{\log_3^{n/3}}})^{-1}] \\
&= n T(1) + n [(\lg n)^{-1} + (\lg \frac{n}{3})^{-1} + (\lg \frac{n}{9})^{-1} +  \cdots + (\lg 3)^{-1}]
\end{split}
\]
Here, we notice that the sequence in the bracket is a harmonic series, 
$\frac{1}{\lg 3}, \frac{1}{\lg 9}, \cdots, \frac{1}{\lg n/9}, \frac{1}{\lg n/3}, \frac{1}{\lg n}$. We can sum them up using the formula learned during the first tutoring session.
\[
\begin{split}
T(n) &= n T(1) + n \sum_{i = 1}^{\log_3^n} \frac{1}{\lg 3^i} \\
&= \Theta(n) + \Theta(n \lg \lg n) \\
&= \Theta(n \lg \lg n)
\end{split}
\]
Finally, the asymptotic runtime is $T(n) = \Theta(n \lg \lg n)$.

\item $T(n) = 4T(n/2) + n^2 \sqrt{n}$

%%% Problem 6.c %%%
\solution

Let's simplify the $f(n)$ part of the equation,
\[
T(n) = 4T(n/2) + n^2 \cdot n^{0.5} = 4T(n/2) + n^{2.5}
\]
This recurrence has the form $T(n) = a T(n/b) + n^c$, so we apply the master theorem. \\
$a = 4$, $b = 2$, and $c = 2.5$. 
Let's do a test, $\frac{a}{b^c} = \frac{4}{2^{2.5}} < 1$, and it corresponds to the case 3 we learned in class. 
Thus, the asymptotic runtime is $T(n) = \Theta (n^{2.5})$.

\item $T(n) = 3T(n/3 - 2) + n/2$

%%% Problem 6.d %%%
\solution

The $-2$ of $T(n/3 - 2)$ can be omitted when we consider the asymptotic runtime because the change $n/3$ is more significant than the constant $-2$ for a large n. Then, the problem can be solved via the master method. \\
$a = 3$, $b = 3$, and $c = 1$. Let's do the test, $\frac{a}{b^c} = \frac{3}{3^{1}} = 1$, which corresponds to the case 2 we learned in class. Thus, the asymptotic runtime is $T(n) = \Theta (n^{\log_b^a} \log n) =  \Theta (n \log n)$. 

\item $T(n) = 2T(n/2) + n / \lg n$

%%% Problem 6.e %%%
\solution

Using the iteration method, we expand the above recurrence as follows,
\[
\begin{split}
T(n) &= 2T(\frac{n}{2}) + n / \lg n \\
&= 2[2T(\frac{n}{4}) + \frac{n}{2}(\lg \frac{n}{2})^{-1}] + n (\lg n)^{-1} = 4T(\frac{n}{4}) + n(\lg \frac{n}{2})^{-1} + n (\lg n)^{-1} \\
& = 4[2T(\frac{n}{8}) + \frac{n}{4}(\lg \frac{n}{4})^{-1}] + n(\lg \frac{n}{2})^{-1} + n (\lg n)^{-1} = 8T(\frac{n}{8}) + n(\lg \frac{n}{4})^{-1} + n(\lg \frac{n}{2})^{-1} + n (\lg n)^{-1}
\end{split}
\]
Now, we can see a clear pattern to represent this recurrence in terms of k,
\[
T(n) = 2^k T(\frac{n}{2^k}) + n \sum_{i=0}^{k-1} (\lg \frac{n}{2^i})^{-1}
\]
We omit proving its correctness via induction because its form is evident through the expansions.
Then, we choose $k = \log_2^n$, which leads to a base case.
\[
\begin{split}
T(n) &= 2^{\log_2^n} T(\frac{n}{2^{\log_2^n}}) + n \sum_{i=0}^{\log_2^n-1} (\lg \frac{n}{2^i})^{-1} \\
&= n T(1) + n [(\lg n)^{-1} +  (\lg \frac{n}{2})^{-1} + (\lg \frac{n}{4})^{-1} + \cdots + (\lg \frac{n}{2^{\log_2^{n/2}}})^{-1}] \\
&= n T(1) + n [(\lg n)^{-1} +  (\lg \frac{n}{2})^{-1} + (\lg \frac{n}{4})^{-1} + \cdots + (\lg 2)^{-1}]
\end{split}
\]
Similar to (b), what's inside the bracket is a harmonic series. The sum of a harmonic series is $\sum_{i=1}^n 1/i \sim \log_e^n + k$.
\[
\begin{split}
T(n) &= n T(1) + n \sum_{i = 1}^{\log_2^n} \frac{1}{\lg 2^i} \\
&= \Theta(n) + \Theta(n \lg \lg n) \\
&= \Theta(n \lg \lg n)
\end{split}
\]
Thus, the asymptotic runtime is $T(n) = \Theta(n \lg \lg n)$.

\item $T(n) = T(n/2) + T(n/4) + T(n/8) + n$

%%% Problem 6.e %%%
\solution

Let's use the substitution method for this problem.

We guess $T(n) = \Theta (n)$, so we need to show $c_1 n \leq T(n) \leq c_2 n$.

\textbf{Proof:}

The lower bound is trivial to prove. Since the original recurrence has a term $+n$, we can choose $c1$ to be 0.5, then $0.5 n \leq T(n/2) + T(n/4) + T(n/8) + n$.

Now, let's prove the upper bound is correct by induction. 
There are three hypotheses:

\begin{math}
  \left\{
    \begin{array}{l}
      T(n/2) \leq cn/2 \\
      T(n/4) \leq cn/4 \\
      T(n/8) \leq cn/8
    \end{array}
  \right.
\end{math}

\[
\begin{split}
T(n) = T(\frac{n}{2}) + T(\frac{n}{4}) + T(\frac{n}{8}) + n 
& \leq c\frac{n}{2} + c\frac{n}{4} + c\frac{n}{8} + n \\
& = (\frac{c}{2} + \frac{c}{4} + \frac{c}{8} + 1)n
\end{split}
\]
We want,
\[
(\frac{c}{2} + \frac{c}{4} + \frac{c}{8} + 1)n \stackrel{?}{\leq} c n
\]
Say we choose $c$ to be 16. $(8 + 4 + 2 + 1)n = 15n \leq 16n$. Thus, the guessed asymptotic runtime $T(n) = \Theta (n)$ is correct.

\end{enumerate}

\end{document}